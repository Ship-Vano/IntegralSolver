\documentclass[12pt, a4paper]{article}

\usepackage[utf8]{inputenc}
\usepackage[T2A]{fontenc}
\usepackage[russian]{babel}
\usepackage[dvips]{graphicx}

\usepackage[oglav,spisok,boldsect,eqwhole,figwhole,hyperref,hyperprint,remarks,greekit]{./style/fn2kursstyle}
\graphicspath{{./style/}{./figures/}}
\usepackage{float}
\usepackage{multirow}
\usepackage{supertabular}
\usepackage{multicol}
\usepackage{hhline}
\usepackage{listings}
\usepackage{color}
\usepackage{adjustbox}
\usepackage{amsmath}
\usepackage{verbatim}
\usepackage{amsfonts}
%\usepackage{mathabx}
\usepackage{graphicx}
\usepackage{subcaption}

\definecolor{dkgreen}{rgb}{0,0.6,0}
\definecolor{gray}{rgb}{0.5,0.5,0.5}
\definecolor{mauve}{rgb}{0.58,0,0.82}

\lstset{frame=tb,
	language=C++,
	aboveskip=1mm,
	belowskip=1mm,
	showstringspaces=false,
	columns=flexible,
	basicstyle={\small},
	numbers=left,
	numberstyle=\tiny\color{gray},
	keywordstyle=\color{red},
	commentstyle=\color{dkgreen},
	stringstyle=\color{mauve},
	breaklines=true,
	breakatwhitespace=true,
	tabsize=2
}
\title{Методы численного решения интегральных уравнений \\ Варианты 5, 16}


%\authorfirst{О.\,Д.~Климов}
%\authorsecond{О.\,Д.~Климов} TODO: прописать команды в style.sty

%\supervisor{С.\,А.~Конев}
\supervisor{ }
\group{ФН2-61Б}
\date{2024}

%\renewcommand{\vec}[1]{\text{\mathversion{bold}${#1}$}}%{\bi{#1}}
\newcommand\thh[1]{\text{\mathversion{bold}${#1}$}}
\renewcommand{\labelenumi}{\theenumi)}
\renewcommand{\labelenumi}{\theenumi)}

\newcommand{\opr}{\textbf{\underline{{Опр.}}}\quad}
\newcommand{\theorem}{\textbf{\underline{{Теор.}}}\quad}
\renewcommand{\phi}{\varphi}
\renewcommand{\k}[1]{\textbf{\textit{#1}}}
\newcommand{\widecheck}[1]{\check{#1}}

\newcounter{mycounter}
\newcommand{\question}[1]{%
	\stepcounter{mycounter}%
	\textbf{\themycounter}.  %
	\textbf{\textit{#1}}
	
}
\newcommand{\down}[1]{\widecheck{#1}}
\newcommand{\pon}[1]{\mathop {#1}\limits^ \circ}
\newcommand{\rusg}{\text{Г}}

\begin{document}
	\maketitle
	\section{Ответы на контрольные вопросы}
	
	\question{При выполнении каких условий интегральное уравнение Фредгольма 2-го рода имеет решение? В каком случае решение является единственным?}	
	Интегральным уравнением Фредгольма 2-го рода называется уравнение следующего вида 
	
	\begin{equation}
		u(x) - \lambda \int_a^b K(x, \xi) u(\xi) d\xi = f(x), \quad x \in [a, b],
		\label{1}
	\end{equation}
	
	где $K$ этого уравнения задана в квадрате $[a, b] \times [a, b]$ и называется ядром.
	
 	Отметим, что для данного уравнения можно поставить задачу Штурама-Лиувилля отыскания собственных функций $\phi_i$ и собственных значений $\lambda_i$. Если ядро вещественно и симметрично, т.е. $K(x, \xi) = K(\xi, x)$, то существует по крайней мере одна собственная функция и одно собственное значение, причем все собственные значения такого оператора действительны, а собственные функции, соответствующие различным собственным значениям, ортогональны.
 
 	Существование решения уравнения Фредгольма 2-го рода и его единственность зависят от параметра $\lambda$. Если $\lambda \neq \lambda_i$ ($\lambda_i$ - собственное число), то уравнение имеет единственное решение. Для симметричного ядра оно может быть представлено в виде разложения Шмидта по системе собственных функций $\varphi_i$.
 	
 	Если $\lambda = \lambda_i$, т.е. параметр совпадает с одним из собственных значений, то при одних правых частях решение вообще не существует, а при других — существует и неединственно.

	
	\bigskip
	\question{Можно ли привести матрицу СЛАУ, получающуюся при использовании метода квадратур, к симметричному виду в случае, если ядро интегрального уравнения является симметричным, т. е. $K(x, s) = K(s, x)$?}
	
	При использовании метода квадратур, получаем СЛАУ в таком виде:
	\begin{equation}
		y_i - \lambda \sum_{k=0}^{N} a_k^N K(x_i, s_k) y_k = f_i, \quad i = 0, \ldots, N.
	\end{equation}
	
	Так как ядро интегрального уравнения симметричное, то $K(x_i, s_k) = K(s_k, x_i)$. Если умножить каждое $i$-е уравнение системы на $a_i^N$:
	\begin{equation}
		a_i^N y_i - \lambda \sum_{k=0}^{N} a_i^N a_k^N K(x_i, s_k) y_k = a_i^N f_i, \quad i = 0, \ldots, N,
	\end{equation}
	то получаем систему с симметричной матрицей.
	
	\bigskip
	\question{Предложите способ контроля точности результата вычислений при использовании метода квадратур.}
	Один из способов контроля точности - это дробление шага и сравнение нормы разности вычисленного значения интеграла с искомой точностью.
	
	\bigskip
	\question{Оцените возможность и эффективность применения методов квадратур, простой итерации и замены ядра при решении интегральных уравнений Вольтерры 2-го рода.}

	\begin{enumerate}
		\item Метод квадратур
		
		 Возможно применение метода квадратур. Получаем СЛАУ
		\[
		y_i - \sum_{k=0}^{N} a_k^N K(x_i, s_k) y_k = f_i
		\]
		с треугольной матрицей, которая решается за один ход метода Гаусса. При этом решение существует и единственно для любого $\lambda$.
		

	
		\item Простой итерации
		
		Итерационные методы, в частности метод простой итерации, также применимы к решению интегрального уравнения Фредгольма 2-го рода. Достаточным условием сходимости метода простой итерации в норме $\|.\|_c$ является выполнение условия
		\[
		|\lambda| \cdot \max_{a \le x \le b} \int_a^b |K(x,s)| ds \le 1.
		\]
		Недостатком метода простой итерации является необходимость приближенного вычисления большого количества интегралов, что может приводить к значительным затратам машинного времени.
		
		
		\item Замена ядра
		
		Ядро интегрального уравнения Фредгольма называется вырожденным, если оно представимо в виде
		\[
		K(x, s) = \sum_{i=1}^{m} \phi_i(x) \psi_i(s),
		\]
		где $\phi_i(x), i = 1, \ldots, m$ - система линейно независимых функций. А вот ядро уравнения Вольтерры вырожденным не бывает. Следовательно, метод замены ядра вырожденным для уравнения Вольтерры 2-го рода не применим.
	\end{enumerate}
	
	
	\bigskip
	\question{Что называют резольвентой ядра интегрального уравнения?}
	
	Резольвентой интегрального уравнения(разрежающим ядром) называется такая функция $R(x, \xi, \lambda)$, что решение уравнения \eqref{1} представляется в виде
	\begin{equation}
		u(x) = f(x) + \lambda \int_a^b R(x, \xi, \lambda) f(\xi) d\xi,
	\end{equation}
	причем $\lambda$ не должна являться собственным числом исходного уравнения.
	
	
	\bigskip
	\question{Почему замену ядра интегрального уравнения вырожденным предпочтительнее осуществлять путем разложения по многочленам Чебышева, а не по формуле Тейлора?}
	
	Полином Чебышева — полином наилучшего приближения функции в данном
	нормированном пространстве, который наилучшим образом аппроксимирует
	функцию на всем исследуемом отрезке. Формула Тейлора записывается в окрестности точки, соответственно, чем дальше находится точка, в которой вычисляется приближенное значение функции, тем больше погрешность аппроксимации. Соответственно, замену ядра интегрального уравнения вырожденным предпочтительнее осуществлять путем разложения по многочленам Чебышева.
	
	В случае полиномиальной аппроксимации 
	\begin{equation}
		||f - L_n||_C \leqslant \dfrac{M_{n + 1}}{(n + 1)!} ||\omega||_C 
	\end{equation}
	Поставив задачу 
	\begin{equation}
		\min_{x_0,\dots,x_n}||\omega||_C - ?,\quad \omega(x) = (x - x_0)\dots(x - x_n)
	\end{equation}
	
	Решением является полином Чебышева
	\begin{equation}
		T_{n + 1}(x) = \dfrac{(b - a)^{n + 1}}{2^{2n + 1}} \cos\Big((n + 1) 
		\arccos{\dfrac{2 x - (b + a)}{b - a}} \Big)
	\end{equation}
	Разложив ядро $K(x, s) = \sum_{n = 0}^{m}\varphi(x)T_{n}(s)$
	получим
	\begin{gather}
		\alpha_{ij} = \int_{a}^{b} T_{j}(s) \varphi_i(s) ds \\
		\beta_i =  \int_{a}^{b} T_{i}(s) f(s) ds
	\end{gather}
	Решая эти интегралы численно получим cледующие оценки:
	\begin{equation}
		|\alpha_{ij} - I_{ij}| \le \dfrac{M_{n+1}}{(n+1)!}\ ||\varphi_i||_C \int_{a}^{b}|T_{j + 1}|dxw	
	\end{equation}
	
	\bigskip
	\question{Какие вы можете предложить методы решения переопределенной системы (5.13), (5.17) помимо введения дополнительно переменной R?}
	
		Для решения переопределенной СЛАУ можно использовать метод наименьших квадратов.
	
	При помощи этого метода коэффициенты аппроксимирующей функции вычисляются таким образом, чтобы среднеквадратичное отклонение экспериментальных данных от найденной аппроксимирующей функции было наименьшим:
	\[
	\sum_{i} \varepsilon_i^2 = \sum_{i} (y_i - f(x_i))^2 \rightarrow min
	\]
	Рассмотрим $A \cdot X = B$. Добавим в уравнение вектор погрешности:
	\[
	A \cdot X = B + \varepsilon
	\]
	Соответствеенно, задача сводится к минимизации квадрата нормы вектора $\varepsilon$:
	\[
	\sum_{i} \varepsilon_i^2 = || \varepsilon ||^2 = \varepsilon^T \cdot \varepsilon \rightarrow min
	\]
	\[
	\varepsilon^T \cdot \varepsilon = (A \cdot X - B)^T (A \cdot X - B) = A^T \cdot X^T \cdot A \cdot X - 2 A^T \cdot X^T \cdot B + B^T \cdot B
	\]
	Для нахождения минимума вычислим частную производную по $X$ и приравнять ее к нулю:
	\[
	\frac{\partial \varepsilon^T \cdot \varepsilon}{\partial X} = 2 A^T \cdot A \cdot X - 2 A^T \cdot B = 0
	\]
	\[
	X = (A^T \cdot A)^{-1} \cdot A^T \cdot B
	\]
	Таким образом, метод наименьших квадратов сводится к нахождению обратной матрицы.

	
	\clearpage
	\section{Дополнительные вопросы}
	
	\question{Метод квадратур. Какие квадратуры Вы использовали в данной лабораторной работе? Какую точность они имеют (порядок, ведущий член погрешности)? Подтвердите расчетами, что такая же точность достигается в Вашей реализации квадратурных формул.}
	
	
	В лабораторной работе была использована квадратурная формула трапеций:
	\[
	\int_{x_i}^{x_{i+1}} f(x) dx \approx \frac{f(x_i) + f(x_{i+1})}{2} h, \quad h = x_{i+1} - x_i,
	\]
	которая имеет оценку локальной погрешности
	\[
	|\psi_{h,i}| \le \frac{1}{12} M_2 h^3.
	\]
	Отсюда получаем оценку погрешности квадратурной формулы трапеций
	\[
	|\psi_h| \le \frac{1}{12} M_2 h^3 n = \frac{1}{12} M_2 h^3 \frac{b - a}{h} = \frac{1}{12} M_2 (b - a) h^2 = O(h^2).
	\]
	Для проверки точности будем использовать следующее интегральное уравнение:
	\[
	u(x) - e \int_1^e \frac{\ln s}{x} u(s) ds = \ln x,
	\]
	имеющее решение вида
	\[
	u(x) = \ln x - \frac{2e}{x}.
	\]
	Для него $f(s) = \frac{e \ln s}{x} \left( \ln s - \frac{2e}{s} \right)$. Следовательно,
	\[
	f''(s) = \frac{2e (3e + s - (2e + s) \ln s)}{s^3 x}.
	\]
	Тогда $M_2 = \max_{1 \le s \le e} \frac{1}{x} \cdot 2e (1 + 3e) = 2e (1 + 3e)$ и достигается при $s = 1$. При $N = 10$ $h = \frac{e - 1}{10}$ и
	\[
	\frac{1}{12} M_2 (b - a) h^2 = \frac{1}{600} (e - 1)^3 (1 + 3e) \approx 0.210415.
	\]
	В реальности абсолютная ошибка равна: $0.116515$.
	
	
	\bigskip
	\question{Критерий останова. Какой критерий останова использовался для метода простой итерации? Вычислите априорную оценку погрешности (приведена в методическом пособии), содержащую множитель q. Проверьте, действительно ли достигаемая погрешность меньше оцениваемой?}
		
	Для метода простой итерации в качестве критериев останова было выбрано условие, что норма разности между последними приближениями меньше некоторого заданного $\varepsilon$:
	\[
	\| u^{(k+1)} - u^{(k)} \| \le \varepsilon.
	\]
	В качестве тестового примера возьмем уравнение
	\[
	u(x) - \frac{1}{2\pi} \int_0^\pi \sin u(s) ds = \cos x, \quad x \in [0, \pi],
	\]
	точное решение которого имеет вид $u(x) = \cos x - \frac{2}{\pi} \sin x$. Для этого примера
	\[
	q = \frac{1}{2\pi} \max_{0 \le x \le \pi} \int_0^\pi |\sin x| ds = \frac{1}{2\pi} \frac{1}{2} \pi^2 \max_{0 \le x \le \pi} \sin x = \frac{\pi}{4} \approx 0.785398 < 1.
	\]
	Было взято 50 узлов.
	
	Для методов типа простой итерации существуют следующие оценки:
	\[
	\| u^{(k)} - u_* \| \le q^k \| u^{(0)} - u_* \|, \quad \| u^{(k)} - u_* \| \le \frac{q^k}{1 - q} \| u^{(0)} - u^{(1)} \|,
	\]
	где $u_*$ - точное решение задачи, $k$ - номер итерации. Будем использовать первую оценку. В качестве начального приближения примем правую часть уравнения $\cos(x)$. Тогда
	\[
	\| u^{(0)} - u_* \| = \left\| \frac{2}{\pi} \sin x \right\| = \frac{2}{\pi}.
	\]
	
	\begin{table}[h!]
		\centering
		\caption{Погрешность метода простой итерации}
		\begin{tabular}{|c|c|c|}
			\hline
			Число итераций & Достигнутая точность & Теоретическая погрешность \\
			\hline
			1 & 0.318205 & 0.5 \\
			2 & 0.15905 & 0.392699 \\
			3 & 0.079489 & 0.308276 \\
			4 & 0.039764 & 0.242237 \\
			5 & 0.0198616 & 0.190292 \\
			6 & 0.00992754 & 0.149424 \\
			7 & 0.0049627 & 0.117205 \\
			8 & 0.0024818 & 0.0918678 \\
			9 & 0.0012409 & 0.072004 \\
			10 & 0.000619629 & 0.056853 \\
			\hline
		\end{tabular}
	\end{table}
		
	\bigskip
	\question{Замена ядра вырожденным. Как меняется погрешность решения с увеличением числа слагаемых в разложении ядра по формуле? Составьте таблицу вида «число слагаемых — достигнутая точность».}
		
		
		
	\bigskip
	\question{Сингулярные уравнения. Установить расчетным путем наименьшее количество точек разбиения окружности, необходимое для получения точного решения сингулярного интегрального уравнения (точное решение можно получить из правой части заменой тригонометрической функции $sin, cos$ и умножением на ±2). Какое количество узлов потребовалось для передачи качественного характера решения?}
		
		
	\bigskip
	\question{Регуляризация. Для решения сингулярного интегрального уравнения в методическом пособии предлагается вводить дополнительную неизвестную. Постройте таблицу, содержащую зависимость величины от числа узлов сетки.}
	
	\clearpage

	\begin{thebibliography}{1}
		\bibitem{1} Галанин М.П., Савенков Е.Б. Методы численного анализа математических моделей. М.: Изд-во МГТУ им. Н.Э. Баумана. 2018. 592 с.
		
	\end{thebibliography}
	
\end{document}